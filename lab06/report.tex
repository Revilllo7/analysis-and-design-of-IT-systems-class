\documentclass[12pt,a4paper]{article}

% ------------------ Pakiety ------------------

\usepackage{multicol}
\usepackage[polish]{babel}
\usepackage[T1]{fontenc}
\usepackage{geometry}
\usepackage{amsmath}
\usepackage{graphicx}
\usepackage{hyperref}
\usepackage[utf8]{inputenc}
\usepackage[T1]{fontenc}
\usepackage{polski}
\usepackage{lmodern}
\usepackage{setspace}
\usepackage{titlesec}
\usepackage{array}
\usepackage[utf8]{inputenc}
\usepackage{polski}
\usepackage{lmodern}
\usepackage{float}


% ------------------ Ustawienia ------------------

\geometry{margin=2.5cm}
\setstretch{1.3}
\titleformat{\section}{\large\bfseries}{\thesection.}{1em}{}
\titleformat{\subsection}{\normalsize\bfseries}{\thesubsection.}{1em}{}
\hypersetup{
    colorlinks=true,
    linkcolor=black,
    urlcolor=blue,
    pdftitle={Diagram przypadków użycia — modelowanie w UML}
}

% ------------------ Dane ------------------

\title{Uniwersytet Gdański
Wydział Matematyki, Fizyki i Informatyki
Instytut Informatyki}
\author{Oliver Gruba, Maciej Nasiadka}

\begin{document}
\maketitle
\begin{table}
    \centering
    \begin{tabular}{|>{\raggedright\arraybackslash}p{0.5\linewidth}|>{\raggedright\arraybackslash}p{0.4\linewidth}|}\hline
         Imię i Nazwisko (nr indeksu)& Oliver Gruba (292583) \\
         & Maciej Nasiadka (292574)\\\hline
         Nazwa uczelni& Uniwersytet Gdański\\\hline
         Kierunek& Informatyka (profil praktyczny)\\\hline
         Prowadzący& dr inż. Stanisław Witkowski\\\hline
         Specjalność& —\\\hline
         Nazwa ćwiczenia& Statyczna struktura systemu informatycznego poprzez tworzenia diagramu\\\hline
         Numer sprawozdania& 3\\\hline
         Data zajęć& 06.11.2025\\\hline
         Data oddania& 12.11.2025\\\hline
         Miejscę na ocenę& \\ \hline
    \end{tabular}
\end{table}

\clearpage

% ------------------ Dokument ------------------

\tableofcontents
\newpage

\section{Diagram klas - notacja i semantyka}

\section{Atrybuty diagramu klas}

\section{Podstawowe relacje stosowane w diagramie klas}

\section{Dziedziczenie}

\section{Przykład 1. (Własny np. materiały)}
todo: update ^

\section{Przykład 2. czytelnia}

\section{Przykład 3. projekt zespołowy}

\section{Wiele diagramów klas w złożonych projektach - przykładowa struktura}

\section{Wnioski (zalety, zastosowania, implementacje, itd)}


\end{document}