\documentclass[a4paper,12pt]{article}
\usepackage[utf8]{inputenc}
\usepackage[polish]{babel}
\usepackage{graphicx}
\usepackage{geometry}
\geometry{margin=2.5cm}
\usepackage{setspace}
\usepackage{hyperref}
\usepackage{float}

\setstretch{1.3}

\begin{document}

\title{\textbf{Diagram czynności UML}}
\author{Imię i nazwisko studenta \\ Kierunek: Informatyka \\ Przedmiot: Analiza i projektowanie systemów informatycznych}
\date{\today}

\maketitle
\tableofcontents
\newpage

\section{Cel diagramów czynności}

Diagram czynności (ang. \textit{Activity Diagram}) jest jednym z podstawowych diagramów behawioralnych UML, służącym do modelowania dynamicznych aspektów systemu. Jego głównym celem jest przedstawienie przepływu sterowania i danych w procesach biznesowych lub operacjach systemowych.

Diagram ten umożliwia:
\begin{itemize}
    \item analizę procesów biznesowych i logiki algorytmów,
    \item wizualizację sekwencji czynności wykonywanych w ramach przypadków użycia,
    \item identyfikację punktów decyzyjnych, rozgałęzień i współbieżności,
    \item ułatwienie komunikacji pomiędzy analitykami, projektantami i użytkownikami.
\end{itemize}

Diagramy czynności stanowią doskonałe narzędzie do opisu scenariuszy przypadków użycia oraz projektowania przepływów pracy (ang. \textit{workflow}) w systemach informatycznych.

\section{Notacja i semantyka}

Na diagramach czynności stosuje się zestaw standardowych elementów UML, które pozwalają odwzorować logikę przepływu sterowania i danych:

\begin{itemize}
    \item \textbf{Węzeł początkowy (Initial Node)} – punkt rozpoczęcia aktywności.
    \item \textbf{Czynność (Action)} – pojedynczy krok lub operacja wykonywana w procesie.
    \item \textbf{Węzeł decyzyjny (Decision Node)} – element rozgałęziający przepływ w zależności od warunków logicznych.
    \item \textbf{Węzeł rozwidlenia (Fork Node)} – rozpoczęcie wykonywania czynności równoległych.
    \item \textbf{Węzeł scalenia (Join Node)} – synchronizacja równoległych przepływów.
    \item \textbf{Węzeł końcowy przepływu (Flow Final Node)} – zakończenie części przepływu.
    \item \textbf{Węzeł końcowy (Activity Final Node)} – zakończenie całej aktywności.
    \item \textbf{Partycje aktywności (Swimlanes)} – wydzielenie czynności realizowanych przez różne role lub systemy.
\end{itemize}

Elementy te łączy się za pomocą przepływów sterowania (\textit{Control Flow}), które określają kolejność wykonywania czynności.

\section{Przykład 1 – Proces tworzenia materiałów}

\textbf{Opis:} Proces przedstawia przygotowanie i publikację materiałów dydaktycznych przez wykładowcę w systemie e-learningowym.

\begin{itemize}
    \item Węzeł początkowy – \textit{Rozpoczęcie przygotowania materiałów}.
    \item Czynność: \textit{Opracowanie treści wykładu}.
    \item Czynność: \textit{Przygotowanie prezentacji}.
    \item Decyzja: \textit{Czy materiał jest kompletny?}
    \begin{itemize}
        \item Jeśli \textbf{tak}: przejście do \textit{Publikacja materiału}.
        \item Jeśli \textbf{nie}: \textit{Poprawa i uzupełnienie treści}.
    \end{itemize}
    \item Po publikacji materiał zostaje udostępniony studentom.
    \item Węzeł końcowy – \textit{Materiał opublikowany}.
\end{itemize}

\begin{figure}[H]
    \centering
    \includegraphics[width=0.8\textwidth]{zmaterialy.png}
    \caption{Diagram czynności – proces tworzenia materiałów dydaktycznych}
\end{figure}

\section{Przykład 2 – System czytelni}

\textbf{Opis:} Diagram przedstawia czynności w systemie czytelni.

\begin{figure}[H]
    \centering
    \includegraphics[width=0.8\textwidth]{resources/diagram_czynnosci_biblioteka.png}
    \caption{Diagram czynności – proces wypożyczania książki w czytelni}
\end{figure}

\section{Przykład 3 – Zadanie zespołowe}

\textbf{Opis:} Diagram ilustruje współpracę zespołu nad realizacją projektu informatycznego.

\begin{itemize}
    \item Czynność 1: \textit{Podział zadań w zespole}.
    \item Rozwidlenie – równoległa praca członków zespołu:
    \begin{itemize}
        \item Programista – \textit{Implementacja modułów}.
        \item Tester – \textit{Testowanie funkcjonalności}.
        \item Dokumentalista – \textit{Tworzenie dokumentacji}.
    \end{itemize}
    \item Scalenie wyników pracy.
    \item Czynność: \textit{Integracja projektu}.
    \item Decyzja: \textit{Czy testy końcowe zakończone sukcesem?}
    \begin{itemize}
        \item Jeśli tak – \textit{Publikacja projektu}.
        \item Jeśli nie – \textit{Poprawa błędów}.
    \end{itemize}
    \item Węzeł końcowy – \textit{Projekt zakończony}.
\end{itemize}

\begin{figure}[H]
    \centering
    \includegraphics[width=0.8\textwidth]{zespol.png}
    \caption{Diagram czynności – realizacja zadania zespołowego}
\end{figure}

\section{Zastosowania diagramów czynności}

Diagramy czynności znajdują szerokie zastosowanie w analizie i projektowaniu systemów informatycznych, w szczególności do:
\begin{itemize}
    \item modelowania procesów biznesowych,
    \item dokumentowania przypadków użycia,
    \item analizy logiki programów i algorytmów,
    \item projektowania przepływów danych i sterowania w systemach,
    \item symulacji współbieżnych działań użytkowników i systemów.
\end{itemize}

\section{Wnioski}

Diagramy czynności stanowią ważny element procesu analizy systemów informatycznych. Pozwalają na:
\begin{itemize}
    \item lepsze zrozumienie przepływu informacji i decyzji w systemie,
    \item identyfikację miejsc potencjalnych błędów lub nieefektywności,
    \item ułatwienie komunikacji pomiędzy zespołami projektowymi i biznesowymi.
\end{itemize}

W kontekście systemów informatycznych diagramy te wspierają planowanie wdrożeń, definiowanie procesów roboczych oraz zapewniają czytelne odwzorowanie dynamiki systemu przed implementacją. Stanowią nieodzowne narzędzie analityka i projektanta oprogramowania.

\end{document}
