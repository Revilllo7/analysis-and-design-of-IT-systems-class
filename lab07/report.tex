\documentclass[12pt,a4paper]{article}

% ------------------ Pakiety ------------------

\usepackage{multicol}
\usepackage[polish]{babel}
\usepackage[T1]{fontenc}
\usepackage{geometry}
\usepackage{amsmath}
\usepackage{graphicx}
\usepackage{hyperref}
\usepackage[utf8]{inputenc}
\usepackage[T1]{fontenc}
\usepackage{polski}
\usepackage{lmodern}
\usepackage{setspace}
\usepackage{titlesec}
\usepackage{array}
\usepackage[utf8]{inputenc}
\usepackage{polski}
\usepackage{lmodern}
\usepackage{float}
\usepackage{listings}
\usepackage{color}
\usepackage{verbatim}

% ------------------ Ustawienia ------------------

\geometry{margin=2.5cm}
\setstretch{1.3}
\titleformat{\section}{\large\bfseries}{\thesection.}{1em}{}
\titleformat{\subsection}{\normalsize\bfseries}{\thesubsection.}{1em}{}
\definecolor{lightgray}{rgb}{.9,.9,.9}
\definecolor{darkgray}{rgb}{.4,.4,.4}
\definecolor{purple}{rgb}{0.65, 0.12, 0.82}

\lstdefinelanguage{JavaScript}{
  keywords={typeof, new, true, false, catch, function, return, null, catch, switch, var, if, in, while, do, else, case, break},
  keywordstyle=\color{blue}\bfseries,
  ndkeywords={class, export, boolean, throw, implements, import, this},
  ndkeywordstyle=\color{darkgray}\bfseries,
  identifierstyle=\color{black},
  sensitive=false,
  comment=[l]{//},
  morecomment=[s]{/*}{*/},
  commentstyle=\color{purple}\ttfamily,
  stringstyle=\color{red}\ttfamily,
  morestring=[b]',
  morestring=[b]''
}

\lstset{
   language=JavaScript,
   backgroundcolor=\color{lightgray},
   extendedchars=true,
   basicstyle=\footnotesize\ttfamily,
   showstringspaces=false,
   showspaces=false,
   numbers=left,
   numberstyle=\footnotesize,
   numbersep=9pt,
   tabsize=2,
   breaklines=true,
   showtabs=false,
   captionpos=b
}

\DeclareUnicodeCharacter{25CF}{$\bullet$}
\DeclareUnicodeCharacter{251C}{\mbox{\kern.23em
  \vrule height2.2exdepth1exwidth.4pt\vrule height2.2ptdepth-1.8ptwidth.23em}}
\DeclareUnicodeCharacter{2500}{\mbox{\vrule height2.2ptdepth-1.8ptwidth.5em}}
\DeclareUnicodeCharacter{2514}{\mbox{\kern.23em
  \vrule height2.2exdepth-1.8ptwidth.4pt\vrule height2.2ptdepth-1.8ptwidth.23em}}

\hypersetup{
    colorlinks=true,
    linkcolor=black,
    urlcolor=blue,
    pdftitle={Statyczna struktura systemu
informatycznego poprzez tworzenia
diagramu}
}



% ------------------ Dane ------------------

\title{Uniwersytet Gdański
Wydział Matematyki, Fizyki i Informatyki
Instytut Informatyki}
\author{Oliver Gruba, Maciej Nasiadka}

\begin{document}
\maketitle
\begin{table}
    \centering
    \begin{tabular}{|>{\raggedright\arraybackslash}p{0.5\linewidth}|>{\raggedright\arraybackslash}p{0.4\linewidth}|}\hline
         Imię i Nazwisko (nr indeksu)& Oliver Gruba (292583) \\
         & Maciej Nasiadka (292574)\\\hline
         Nazwa uczelni& Uniwersytet Gdański\\\hline
         Kierunek& Informatyka (profil praktyczny)\\\hline
         Prowadzący& dr inż. Stanisław Witkowski\\\hline
         Nazwa ćwiczenia& Dynamiczny aspekt systemu informatycznego poprzez tworzenie behawioralnego diagramu czynności\\\hline
         Numer sprawozdania& 4\\\hline
         Data zajęć& 13.11.2025\\\hline
         Data oddania& 19.11.2025\\\hline
         Miejscę na ocenę& \\ \hline
    \end{tabular}
\end{table}

\clearpage

% ------------------ Dokument ------------------

\tableofcontents
\newpage

\section{Cel diagramów czynności}

Diagram czynności (ang. \textit{Activity Diagram}) jest jednym z podstawowych diagramów behawioralnych UML, służącym do modelowania dynamicznych aspektów systemu. Jego celem jest przedstawienie:

\begin{itemize}
    \item przepływu działań wykonywanych w systemie,

    \item kolejności operacji,

    \item warunków decyzyjnych,

    \item równoległego wykonywania czynności,

    \item przepływu informacji oraz sterowania,

    \item interakcji użytkownika z systemem na poziomie procesów,

    \item reakcji systemu na zdarzenia wewnętrzne i zewnętrzne (np. zadziałanie przypomnień, wysłanie powiadomień).
\end{itemize}

Diagramy czynności stanowią doskonałe narzędzie do opisu scenariuszy przypadków użycia oraz projektowania przepływów pracy (ang. \textit{workflow}) w systemach informatycznych. Pozwalają nie tylko opisać zachowanie systemu, ale też je przeanalizować pod względami jak:

\begin{itemize}
    \item logiki biznesowej,

    \item przebiegu scenariuszy użytkownika (use cases),

    \item punktów optymalizacji procesów,

    \item potencjalnych błędów sekwencji lub niejednożnaczności funkcjonalnych,

    \item miejsc wymagajacych synchronizacji lub dodatkowej walidacji.
\end{itemize}

W projektach zespołowych diagramy czynności pełnią również rolę komunikacyjną, międzyinnymi pozwalając członkom zespołu szybko zrozumieć procesy, nawet jeśli nie są sztywno odpowiedzialni za ich implementację.

\section{Notacja i semantyka}

Na diagramach czynności stosuje się zestaw standardowych elementów UML, które pozwalają odwzorować logikę przepływu sterowania i danych:

\subsection{Elementy podstawowe}

\begin{itemize}
    \item Początek (Initial Node) — symbol wznowienia działania, zwykle jako czarne koło.

    \item Czynność (Action Node) — pojedynczy krok, operacja wykonywana przez system lub użytkownika.

    \item Przejście (Transition) — strzałka określająca przepływ sterowania między czynnościami.

    \item Zakończenie (Activity Final Node) — punkt kończący cały proces.

    \item Flow Final — punkt kończący tylko jedną z gałęzi przepływu.
\end{itemize}

\begin{figure}[H]
    \centering
    \includegraphics[width=0.75\linewidth]{resources/activity_diagram/Podstawowe elementy diagramu.png}
    \caption{Graficzne przedstawienie podstawowych elementów diagramu: Initial node, Action node, Transition, Activity final node i Flow final}
    \label{fig:elementy_podstawowe}
\end{figure}

\subsection{Elementy kontrolne}

\begin{itemize}
    \item Węzeł decyzyjny (Decision Node) — wybór jednej z możliwych ścieżek na podstawie warunku.

    \item Węzeł łączenia (Merge Node) — ponowne scalenie alternatywnych ścieżek.
\end{itemize}

\begin{figure}[H]
    \centering
    \includegraphics[width=0.75\linewidth]{resources/activity_diagram/Elementy kontrolne diagramu.png}
    \caption{Graficzne przedstawienie elementów kontrolnych diagramu: Decision node i Merge node}
    \label{fig:elementy_kontrolne}
\end{figure}

\subsection{Równoległość}

\begin{itemize}
    \item Fork Node — rozdzielenie jednego przepływu na wiele równoległych.

    \item Join Node — synchronizacja kilku równoległych ścieżek w jedną.
\end{itemize}

\begin{figure}[H]
    \centering
    \includegraphics[width=0.75\linewidth]{resources/activity_diagram/Równoległości diagramu.png}
    \caption{Graficzne przedstawienie podstawowych elementów diagramu: Fork node i Join node}
    \label{fig:równoległości_diagramu}
\end{figure}

\subsection{Obiekty i dane}

\begin{itemize}
    \item Object Node — reprezentuje obiekt, dokument lub dane przechodzące między czynnościami.

    \item Swimlane (tor aktywności) — podział diagramu na obszary reprezentujące odpowiedzialność uczestników (np. Użytkownik, System, Moduł Powiadomień).
\end{itemize}

\begin{figure}[H]
    \centering
    \includegraphics[width=0.75\linewidth]{resources/activity_diagram/Obiekty i dane diagramu.png}
    \caption{Graficzne przedstawienie podstawowych elementów diagramu: Object node i Swimlane}
    \label{fig:obiekty_i_dane_diagramu}
\end{figure}

\subsection{Semantyka}

\begin{itemize}
    \item Przepływ odbywa się sekwencyjnie lub równolegle w zależności od struktury.

    \item Czynność wykonuje się w atomicznie (w całości).

    \item Warunki decyzyjne muszą być wzajemnie rozłączne, chyba że diagram zakłada inaczej.

    \item Równoległość wymaga jawnej synchronizacji Join Node, aby uniknąć niespójności.
\end{itemize}

Elementy te łączy się za pomocą przepływów sterowania (\textit{Control Flow}), które określają kolejność wykonywania czynności.

\section{Przykład 1 - Proces kupowania pojazdu samochodowego}

\begin{figure}[H]
    \centering
    \includegraphics[width=0.75\linewidth]{resources/teacher/teacher_activity_diagram.png}
    \caption{Diagram czynności - przedstawia proces zamawiania pojazdu przez użytkownika}
    \label{fig:activity_rent_car}
\end{figure}

Proces przedstawia system zamawiania pojazdów samochodowych przez użytkownika.

\begin{itemize}
    \item Initial node - \textit{''Użytkownik na stronie''}.
    \item Action node: \textit{''Uruchomienie koszyka''}.
    \item Action node: \textit{''Wyświetlenie koszyka''}.
    \item Merge node:
    \begin{itemize}
        \item \textit{''Wyświetlenie koszyka''}: przejście do \textit{Decision Node ''Czy zmiany w koszyku?''}.
        \item \textit{''Aktualizacja danych w koszyku''}: przejście do \textit{Decision Node ''Czy zmiany w koszyku?''}.
    \end{itemize}
    
    \item Decision node: \textit{Czy zmiany w koszyku?}:
    \begin{itemize}
        \item Jeśli \textbf{tak}: przejście do \textit{Action node: ''Zamiana produktów w koszyku''}.
        \begin{itemize}
            \item Action node: \textit{''Aktualizacja danych w koszyku'' przejście do Merge node}

            \item Jeśli \textbf{nie}: przejście do \textit{Action node: ''Zatwierdzenie koszyka''}.
        \end{itemize}
    \end{itemize}
    
    \item Decision node: \textit{''Czy suma zamówienia > 0?''}
    \begin{itemize}
        \item Jeśli \textbf{tak}: przejście do \textit{Action node: ''Wyświetlenie formularza z danymi klienta''}.
        
        \item Jeśli \textbf{nie}: przejście do \textit{Action node: ''Wyświetlenie skróconego formularza''}.
    \end{itemize}
    
    \item Merge node:
    \begin{itemize}
        \item \textit{''Wyświetlenie formularza z danymi klienta''}: przejście do \textit{Decision Node ''Wypełnianie formularza''}.
        
        \item \textit{''Wyświetlanie skróconego formularza''}: przejście do \textit{Decision Node ''Wypełnianie formularza''}.
        
        \item \textit{''Wyświetlanie listy pól do poprawy''}: przejście do \textit{Decision Node ''Wypełnianie formularza''}.
        
    \end{itemize}
    \item Action node: \textit{''Wypełnianie formularza'}

    \item Action node: \textit{''Walidacja danych''}

    \item Decision node: \textit{''Dane poprawne?''}
    \begin{itemize}
        \item Jeśli \textbf{tak}: przejście do \textit{Action node: ''Zapisanie formularza''}.
        
        \item Jeśli \textbf{nie}: przejście do \textit{Action node: ''Wyświetlanie listy pól do poprawy''}.
    \end{itemize}

    \item Action node: \textit{''Zapisanie formularza''}

    \item Fork node: \textit{Action node: ''Zapisanie formularza'' przechodzi do}:
    \begin{itemize}
        \item Action node: \textit{''Utworzenie konta użytkownika''}.
        
        \item Action node: \textit{''Rezerwacja pojazdu''}.
    \end{itemize}

    \item Join node: \textit{Action nodes ''Utworzenie konta użytkownika'' i ''Rezerwacja pojazdu''}  łączą się do:
    \begin{itemize}
        \item Activity final node: \textit{''Utworzone konto''}.
    \end{itemize}
    
    \item Węzeł końcowy - \textit{''Utworzone konto''}.
\end{itemize}



\section{Przykład 2 - System czytelni}

\textbf{Opis:} Diagram przedstawia czynności w systemie czytelni.


\section{Przykład 3 - Zadanie zespołowe}
\label{kairo_diagram}

\textbf{Opis:} Diagram ilustruje czynności w projekcie Kairo Habit App.

\begin{figure}[H]
    \centering
    \includegraphics[width=0.8\textwidth]{resources/library/diagram_czynnosci_biblioteka.png}
    \caption{Diagram czynności Kairo Habit App}
\end{figure}

Dla tego systemu przykład diagramu czynności pokazuje scenariusz użytkownika ustawiającego nowy nawyk z przypomnieniem.
Taki przypadek użycia łączy interakcję człowieka z automatycznymi procesami wewnątrz systemu.

\section{Zastosowania diagramów czynności}

Diagramy czynności znajdują szerokie zastosowanie w analizie i projektowaniu systemów informatycznych, w szczególności do:

\subsection{Analiza i modelowanie wymagań}

\begin{itemize}
    \item Pozwalają dokładnie opisać scenariusze użytkownika.

    \item Umożliwiają wychwycenie brakujących kroków lub warunków.
\end{itemize}

\subsection{Projektowanie logiki biznesowej}

\begin{itemize}
    \item Wizualizują przepływy danych i operacji.

    \item Pomagają projektantom określić odpowiedzialności modułów.
\end{itemize}

\subsection{Komunikacja w zespole}

\begin{itemize}
    \item Ułatwiają współpracę między analitykami, projektantami i programistami.

    \item Eliminują nieporozumienia wynikające z interpretacji słownego opisu.
\end{itemize}

\subsection{Testowanie}

\begin{itemize}
    \item Służą jako podstawa do tworzenia przypadków testowych.

    \item Pozwalają odwzorować ścieżki alternatywne i sytuacje wyjątkowe.
\end{itemize}

\subsection{Optymalizacja procesów}
\begin{itemize}
    \item Pokazują miejsca, gdzie można wprowadzić asynchroniczność, skrócić proces, lub dodać walidację.
\end{itemize}

\section{Wnioski}

Diagramy czynności stanowią ważny element procesu analizy systemów informatycznych. Pozwalają na:
\begin{itemize}
    \item pokazanie, jak system zachowuje się w odpowiedzi na działania użytkownika,

    \item wskazanie zależności czasowych i logicznych między operacjami,

    \item uchwycenie procesów współbieżnych,

    \item weryfikacja poprawności zaprojektowanych funkcji jeszcze przed implementacją.
\end{itemize}

W kontekście projektów zespołowych, takich jak opisywana aplikacja do nawyków (referencja: \ref{kairo_diagram}):
\begin{itemize}
    \item diagramy czynności porządkują i formalizują pracę nad modułami,

    \item zwiększają spójność projektową,

    \item redukują ryzyko błędnych interpretacji,

    \item wspierają testowanie i wdrożenie.
\end{itemize}

Ostatecznie, w dobrze zaprojektowanym systemie informatycznym diagramy czynności pozwalają na klarowne odwzorowanie procesów biznesowych oraz zapewniają, że implementacja odpowiada rzeczywistym potrzebom użytkownika i założeniom funkcjonalnym.

\end{document}