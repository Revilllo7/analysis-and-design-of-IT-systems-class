\documentclass[12pt,a4paper]{article}

% ------------------ Pakiety ------------------

\usepackage{multicol}
\usepackage[polish]{babel}
\usepackage[T1]{fontenc}
\usepackage{geometry}
\usepackage{amsmath}
\usepackage{graphicx}
\usepackage{hyperref}
\usepackage[utf8]{inputenc}
\usepackage[T1]{fontenc}
\usepackage{polski}
\usepackage{lmodern}
\usepackage{setspace}
\usepackage{titlesec}
\usepackage{array}
\usepackage[utf8]{inputenc}
\usepackage{polski}
\usepackage{lmodern}
\usepackage{float}
\usepackage{listings}
\usepackage{color}
\usepackage{verbatim}

% ------------------ Ustawienia ------------------

\geometry{margin=2.5cm}
\setstretch{1.3}
\titleformat{\section}{\large\bfseries}{\thesection.}{1em}{}
\titleformat{\subsection}{\normalsize\bfseries}{\thesubsection.}{1em}{}
\definecolor{lightgray}{rgb}{.9,.9,.9}
\definecolor{darkgray}{rgb}{.4,.4,.4}
\definecolor{purple}{rgb}{0.65, 0.12, 0.82}

\lstdefinelanguage{JavaScript}{
  keywords={typeof, new, true, false, catch, function, return, null, catch, switch, var, if, in, while, do, else, case, break},
  keywordstyle=\color{blue}\bfseries,
  ndkeywords={class, export, boolean, throw, implements, import, this},
  ndkeywordstyle=\color{darkgray}\bfseries,
  identifierstyle=\color{black},
  sensitive=false,
  comment=[l]{//},
  morecomment=[s]{/*}{*/},
  commentstyle=\color{purple}\ttfamily,
  stringstyle=\color{red}\ttfamily,
  morestring=[b]',
  morestring=[b]''
}

\lstset{
   language=JavaScript,
   backgroundcolor=\color{lightgray},
   extendedchars=true,
   basicstyle=\footnotesize\ttfamily,
   showstringspaces=false,
   showspaces=false,
   numbers=left,
   numberstyle=\footnotesize,
   numbersep=9pt,
   tabsize=2,
   breaklines=true,
   showtabs=false,
   captionpos=b
}

\DeclareUnicodeCharacter{25CF}{$\bullet$}
\DeclareUnicodeCharacter{251C}{\mbox{\kern.23em
  \vrule height2.2exdepth1exwidth.4pt\vrule height2.2ptdepth-1.8ptwidth.23em}}
\DeclareUnicodeCharacter{2500}{\mbox{\vrule height2.2ptdepth-1.8ptwidth.5em}}
\DeclareUnicodeCharacter{2514}{\mbox{\kern.23em
  \vrule height2.2exdepth-1.8ptwidth.4pt\vrule height2.2ptdepth-1.8ptwidth.23em}}

\hypersetup{
    colorlinks=true,
    linkcolor=black,
    urlcolor=blue,
    pdftitle={Statyczna struktura systemu
informatycznego poprzez tworzenia
diagramu}
}



% ------------------ Dane ------------------

\title{Uniwersytet Gdański
Wydział Matematyki, Fizyki i Informatyki
Instytut Informatyki}
\author{Oliver Gruba, Maciej Nasiadka}

\begin{document}
\maketitle
\begin{table}
    \centering
    \begin{tabular}{|>{\raggedright\arraybackslash}p{0.5\linewidth}|>{\raggedright\arraybackslash}p{0.4\linewidth}|}\hline
         Imię i Nazwisko (nr indeksu)& Oliver Gruba (292583) \\
         & Maciej Nasiadka (292574)\\\hline
         Nazwa uczelni& Uniwersytet Gdański\\\hline
         Kierunek& Informatyka (profil praktyczny)\\\hline
         Prowadzący& dr inż. Stanisław Witkowski\\\hline
         Specjalność& -\\\hline
         Nazwa ćwiczenia& Dynamiczny aspekt systemu informatycznego poprzez tworzenie behawioralnego diagramu czynności\\\hline
         Numer sprawozdania& 4\\\hline
         Data zajęć& 13.11.2025\\\hline
         Data oddania& 19.11.2025\\\hline
         Miejscę na ocenę& \\ \hline
    \end{tabular}
\end{table}

\clearpage

% ------------------ Dokument ------------------

\tableofcontents
\newpage

\section{Cel diagramów czynności}

Diagram czynności (ang. \textit{Activity Diagram}) jest jednym z podstawowych diagramów behawioralnych UML, służącym do modelowania dynamicznych aspektów systemu. Jego głównym celem jest przedstawienie przepływu sterowania i danych w procesach biznesowych lub operacjach systemowych.

Diagram ten umożliwia:
\begin{itemize}
    \item analizę procesów biznesowych i logiki algorytmów,
    \item wizualizację sekwencji czynności wykonywanych w ramach przypadków użycia,
    \item identyfikację punktów decyzyjnych, rozgałęzień i współbieżności,
    \item ułatwienie komunikacji pomiędzy analitykami, projektantami i użytkownikami.
\end{itemize}

Diagramy czynności stanowią doskonałe narzędzie do opisu scenariuszy przypadków użycia oraz projektowania przepływów pracy (ang. \textit{workflow}) w systemach informatycznych.

\section{Notacja i semantyka}

Na diagramach czynności stosuje się zestaw standardowych elementów UML, które pozwalają odwzorować logikę przepływu sterowania i danych:

\begin{itemize}
    \item \textbf{Węzeł początkowy (Initial Node)} - punkt rozpoczęcia aktywności.
    \item \textbf{Czynność (Action)} - pojedynczy krok lub operacja wykonywana w procesie.
    \item \textbf{Węzeł decyzyjny (Decision Node)} - element rozgałęziający przepływ w zależności od warunków logicznych.
    \item \textbf{Węzeł rozwidlenia (Fork Node)} - rozpoczęcie wykonywania czynności równoległych.
    \item \textbf{Węzeł scalenia (Join Node)} - synchronizacja równoległych przepływów.
    \item \textbf{Węzeł końcowy przepływu (Flow Final Node)} - zakończenie części przepływu.
    \item \textbf{Węzeł końcowy (Activity Final Node)} - zakończenie całej aktywności.
    \item \textbf{Partycje aktywności (Swimlanes)} - wydzielenie czynności realizowanych przez różne role lub systemy.
\end{itemize}

\begin{figure}[H]
    \centering
    \includegraphics[width=1\linewidth]{Activity-Diagram-Notations.jpg}
    \caption{Schemat przedstawienia graficznych elementów diagramu czynności}
    \label{fig:activity_notations}
\end{figure}

Elementy te łączy się za pomocą przepływów sterowania (\textit{Control Flow}), które określają kolejność wykonywania czynności.

\section{Przykład 1 - Proces kupowania pojazdu samochodowego}

\begin{figure}[H]
    \centering
    \includegraphics[width=0.75\linewidth]{resources/teacher_activity_diagram.png}
    \caption{Diagram czynności - przedstawia proces zamawiania pojazdu przez użytkownika}
    \label{fig:activity_rent_car}
\end{figure}

Proces przedstawia ---.

\begin{itemize}
    \item Węzeł początkowy - \textit{---}.
    \item Czynność: \textit{---}.
    \item Czynność: \textit{---}.
    \item Decyzja: \textit{---}
    \begin{itemize}
        \item Jeśli \textbf{tak}: przejście do \textit{---}.
        \item Jeśli \textbf{nie}: \textit{---}.
    \end{itemize}
    \item ---.
    \item Węzeł końcowy - \textit{---}.
\end{itemize}


\section{Przykład 2 - System czytelni}

\textbf{Opis:} Diagram przedstawia czynności w systemie czytelni.


\section{Przykład 3 - Zadanie zespołowe}

\textbf{Opis:} Diagram ilustruje czynności w projekcie Kairo Habit App.

\begin{figure}[H]
    \centering
    \includegraphics[width=0.8\textwidth]{resources/diagram_czynnosci_biblioteka.png}
    \caption{Diagram czynności Kairo Habit App}
\end{figure}

\section{Zastosowania diagramów czynności}

Diagramy czynności znajdują szerokie zastosowanie w analizie i projektowaniu systemów informatycznych, w szczególności do:
\begin{itemize}
    \item modelowania procesów biznesowych,
    \item dokumentowania przypadków użycia,
    \item analizy logiki programów i algorytmów,
    \item projektowania przepływów danych i sterowania w systemach,
    \item symulacji współbieżnych działań użytkowników i systemów.
\end{itemize}

\section{Wnioski}

Diagramy czynności stanowią ważny element procesu analizy systemów informatycznych. Pozwalają na:
\begin{itemize}
    \item lepsze zrozumienie przepływu informacji i decyzji w systemie,
    \item identyfikację miejsc potencjalnych błędów lub nieefektywności,
    \item ułatwienie komunikacji pomiędzy zespołami projektowymi i biznesowymi.
\end{itemize}

W kontekście systemów informatycznych diagramy te wspierają planowanie wdrożeń, definiowanie procesów roboczych oraz zapewniają czytelne odwzorowanie dynamiki systemu przed implementacją. Stanowią nieodzowne narzędzie analityka i projektanta oprogramowania.

\end{document}