\documentclass[12pt,a4paper]{article}

% ------------------ Pakiety ------------------

\usepackage{multicol}
\usepackage[polish]{babel}
\usepackage[T1]{fontenc}
\usepackage{geometry}
\usepackage{amsmath}
\usepackage{graphicx}
\usepackage{hyperref}
\usepackage[utf8]{inputenc}
\usepackage[T1]{fontenc}
\usepackage{polski}
\usepackage{lmodern}
\usepackage{setspace}
\usepackage{titlesec}
\usepackage{array}
\usepackage[utf8]{inputenc}
\usepackage{polski}
\usepackage{lmodern}
\usepackage{float}
\usepackage{listings}
\usepackage{color}
\usepackage{verbatim}

% ------------------ Ustawienia ------------------

\geometry{margin=2.5cm}
\setstretch{1.3}
\titleformat{\section}{\large\bfseries}{\thesection.}{1em}{}
\titleformat{\subsection}{\normalsize\bfseries}{\thesubsection.}{1em}{}
\definecolor{lightgray}{rgb}{.9,.9,.9}
\definecolor{darkgray}{rgb}{.4,.4,.4}
\definecolor{purple}{rgb}{0.65, 0.12, 0.82}

\lstdefinelanguage{JavaScript}{
  keywords={typeof, new, true, false, catch, function, return, null, catch, switch, var, if, in, while, do, else, case, break},
  keywordstyle=\color{blue}\bfseries,
  ndkeywords={class, export, boolean, throw, implements, import, this},
  ndkeywordstyle=\color{darkgray}\bfseries,
  identifierstyle=\color{black},
  sensitive=false,
  comment=[l]{//},
  morecomment=[s]{/*}{*/},
  commentstyle=\color{purple}\ttfamily,
  stringstyle=\color{red}\ttfamily,
  morestring=[b]',
  morestring=[b]''
}

\lstset{
   language=JavaScript,
   backgroundcolor=\color{lightgray},
   extendedchars=true,
   basicstyle=\footnotesize\ttfamily,
   showstringspaces=false,
   showspaces=false,
   numbers=left,
   numberstyle=\footnotesize,
   numbersep=9pt,
   tabsize=2,
   breaklines=true,
   showtabs=false,
   captionpos=b
}

\DeclareUnicodeCharacter{25CF}{$\bullet$}
\DeclareUnicodeCharacter{251C}{\mbox{\kern.23em
  \vrule height2.2exdepth1exwidth.4pt\vrule height2.2ptdepth-1.8ptwidth.23em}}
\DeclareUnicodeCharacter{2500}{\mbox{\vrule height2.2ptdepth-1.8ptwidth.5em}}
\DeclareUnicodeCharacter{2514}{\mbox{\kern.23em
  \vrule height2.2exdepth-1.8ptwidth.4pt\vrule height2.2ptdepth-1.8ptwidth.23em}}

\hypersetup{
    colorlinks=true,
    linkcolor=black,
    urlcolor=blue,
    pdftitle={Interakcja procesu programu informatycznego w czasie poprzez tworzenie diagramu sekwencji}
}



% ------------------ Dane ------------------

\title{Uniwersytet Gdański
Wydział Matematyki, Fizyki i Informatyki
Instytut Informatyki}
\author{Oliver Gruba, Maciej Nasiadka}

\begin{document}
\maketitle
\begin{table}
    \centering
    \begin{tabular}{|>{\raggedright\arraybackslash}p{0.5\linewidth}|>{\raggedright\arraybackslash}p{0.4\linewidth}|}\hline
         Imię i Nazwisko (nr indeksu)& Oliver Gruba (292583) \\
         & Maciej Nasiadka (292574)\\\hline
         Nazwa uczelni& Uniwersytet Gdański\\\hline
         Kierunek& Informatyka (profil praktyczny)\\\hline
         Prowadzący& dr inż. Stanisław Witkowski\\\hline
         Nazwa ćwiczenia& Interakcja procesu programu informatycznego w czasie poprzez tworzenie diagramu sekwencji\\\hline
         Numer sprawozdania& 6\\\hline
         Data zajęć& 27.11.2025\\\hline
         Data oddania& 03.12.2025\\\hline
         Miejscę na ocenę& \\ \hline
    \end{tabular}
\end{table}
\clearpage

% ------------------ Dokument ------------------

\tableofcontents

\clearpage

\section{Procedura uruchamiania diagramu sekwencji}

Diagram sekwencji jest narzędziem dynamicznym, służącym do odwzorowania zachowania systemu w czasie. Aby poprawnie zinterpretować, uruchomić i przeanalizować diagram sekwencji, konieczne jest zastosowanie określonej procedury. Pozwala ona na konsekwentne prześledzenie przepływu komunikatów pomiędzy aktorami i obiektami, a także na analizę logiki procesów biznesowych.

\subsection{Identyfikacja aktorów i obiektów uczestniczących w procesie}
\begin{itemize}
    \item Określenie, które elementy systemu biorą udział w analizowanej interakcji.
    \item Wyróżnienie aktorów zewnętrznych (np. użytkownik, system nadrzędny).
    \item Wyodrębnienie obiektów wewnętrznych systemu (np. podsystemy).
    \item Ustalenie zależności między elementami oraz zakresu ich odpowiedzialności.\newline
    \qquad Np. rozróżnienie międy komunikacją między frontendem, backendem i bazą danych.
\end{itemize}

\subsection{Określenie scenariusza bazowego}
\begin{itemize}
    \item Zdefiniowanie kolejnych kroków przewidywanej interakcji.
    \item Zapisanie scenariusza w sposób sekwencyjny, od pierwszego do ostatniego kroku.\newline
    \qquad Gdzie idąc kolejno w dół diagrmau pokazujemy elementy, które wykonują się później w systemie (zachowując hierarchie kolejności procesów).
    \item Uwzględnienie logiki biznesowej bez wprowadzania wyjątków.
\end{itemize}

\subsection{Dodanie scenariuszy alternatywnych}
\begin{itemize}
    \item Identyfikacja możliwych punktów decyzyjnych w procesie.
    \item Określenie sytuacji wyjątkowych (np. błędy, odmowy dostępu, brak środków).
    \item Uporządkowanie alternatyw przy użyciu konstrukcji \texttt{alt} i \texttt{opt}.
    \item Zapewnienie kompletności scenariuszy poprzez uwzględnienie wszystkich możliwych ścieżek.
\end{itemize}

\subsection{Modelowanie komunikacji między obiektami}
\begin{itemize}
    \item Przedstawienie wywołań metod, wymiany komunikatów, sygnałów asynchronicznych.
    \item Określenie, czy komunikacja jest synchroniczna (blokująca), czy asynchroniczna.
    \item Uporządkowanie komunikatów od góry do dołu, zgodnie z czasem ich wystąpienia.
\end{itemize}

\subsection{Analiza i walidacja poprawności diagramu}
\begin{itemize}
    \item Sprawdzenie spójności diagramu z rzeczywistym przebiegiem operacji.
    \item Weryfikacja kompletności procesu oraz zgodności ze specyfikacją wymagań.
    \item Ocena poprawności połączeń pomiędzy liniami życia.
    \item Eliminacja błędów logicznych oraz niepotrzebnych elementów.
\end{itemize}

\section{Cel tworzenia diagramu sekwencji}

Diagram sekwencji jest kluczowym narzędziem analizy dynamicznej systemu, pozwalającym na precyzyjne odwzorowanie logiki przepływu zdarzeń w czasie. Jego głównym celem jest wyjaśnienie sposobu współpracy obiektów i aktorów podczas realizacji konkretnej funkcji systemu.

\subsection{Przedstawienie dynamiki systemu w czasie}
\begin{itemize}
    \item Umożliwia modelowanie zachowania systemu w odpowiedzi na działania aktora.
    \item Ukazuje kolejność operacji oraz reakcje obiektów.
    \item Pozwala obserwować logikę sterowania procesem w sposób chronologiczny.
\end{itemize}

\subsection{Dokumentacja interakcji między obiektami}
\begin{itemize}
    \item Wskazuje sposób komunikacji pomiędzy komponentami systemu.
    \item Wyjaśnia, jakie metody lub komunikaty są wywoływane.
    \item Umożliwia zrozumienie odpowiedzialności poszczególnych elementów systemu.
\end{itemize}

\subsection{Wspomaganie projektowania architektury}
\begin{itemize}
    \item Ułatwia podział systemu na moduły oraz określenie ich odpowiedzialności.
    \item Pomaga wykryć przeciążone lub zbędne elementy logiki.
    \item Przedstawia miejsca, w których występuje zależność czasowa.
\end{itemize}

\subsection{Analiza warunków wyjątkowych i przypadków alternatywnych}
\begin{itemize}
    \item Pozwala wykryć potencjalne błędy na wczesnym etapie projektowania.
    \item Ułatwia dokumentowanie sytuacji, w których logika procesu się rozgałęzia.
    \item Formalizuje decyzje i punkty kontrolne w systemie.
\end{itemize}

\subsection{Wspieranie komunikacji w zespole projektowym}
\begin{itemize}
    \item Stanowi wspólny język pomiędzy analitykami, programistami i testerami.
    \item Ujednolica sposób interpretacji funkcjonalności systemu.
    \item Jest przydatny w trakcie przeglądów technicznych oraz analiz kodu.
\end{itemize}

\section{Linie życia aktorów i obiektów}

Linie życia (\textit{lifelines}) stanowią podstawowy element diagramu sekwencji i służą do reprezentacji istnienia aktora lub obiektu w czasie. Każdy uczestnik interakcji posiada własną linię życia, która biegnie pionowo od góry diagramu w dół.

\subsection{Definicja linii życia}
\begin{itemize}
    \item Jest to pionowa linia symbolizująca czas istnienia obiektu w trakcie wykonywania procesu.
    \item Linia zaczyna się w momencie, w którym obiekt pojawia się w systemie, a kończy w chwili zakończenia interakcji.
    \item Może reprezentować zarówno aktorów zewnętrznych, jak i obiekty systemowe.
\end{itemize}

\subsection{Reprezentacja aktywności obiektu}
\begin{itemize}
    \item Aktywność często oznaczana jest prostokątnym paskiem (tzw. \textit{activation bar}).
    \item Pasek aktywności pokazuje okres, w którym obiekt wykonuje operację lub przetwarza dane.
    \item Dłuższe okresy aktywności mogą oznaczać intensywną logikę obliczeniową lub oczekiwanie na odpowiedź.
\end{itemize}

\subsection{Komunikaty wpływające na linię życia}
\begin{itemize}
    \item Komunikaty synchroniczne rozpoczynają aktywność obiektu (np. wywołanie metody).
    \item Komunikaty zwrotne kończą aktywność lub sygnalizują rezultat operacji.
    \item Komunikaty asynchroniczne mogą nie wymagać aktywności zwrotnej.
\end{itemize}

\subsection{Zależności czasowe}
\begin{itemize}
    \item Im niżej umieszczony komunikat, tym później występuje w czasie.
    \item Linie życia pozwalają analizować, jak długo trwa przetwarzanie poszczególnych operacji.
    \item Dzięki nim możliwe jest wskazanie wąskich gardeł lub punktów blokujących system.
\end{itemize}

\subsection{Interpretacja zakończenia życia obiektu}
\begin{itemize}
    \item Zakończenie linii życia często przedstawione jest za pomocą symbolu \textit{X}.
    \item Oznacza moment, w którym obiekt jest usuwany lub kończy działanie w procesie.
    \item W przypadku aktorów zakończenie linii życia sygnalizuje zakończenie interakcji użytkownika z systemem.
\end{itemize}

\section{Zadanie 1 - przykład z materiałów.}

\begin{figure}[H]
    \centering
    \includegraphics[width=0.65\linewidth]{resources/teacher/sekwencja_teacher.png}
    \caption{Diagram sekwencji z materiałów od prowadzącego. Przedstawiający użycie bankomatu przez użytkownika i przebieg aktualizacji systemu serwera bankowego.}
    \label{fig:teacher_sequence_diagram}
\end{figure}

\subsection{Rozpoczęcie interakcji: uwierzytelnienie użytkownika}

Proces rozpoczyna się, gdy Użytkownik umieszcza kartę w bankomacie:
\begin{itemize}
    \item \textbf{Użytkownik} wykonuje akcję \textit{Insert Card}.
    \item \textbf{ATM} odpowiada żądaniem wprowadzenia numeru PIN \textit{Request PIN}.
    \item \textbf{Użytkownik} wprowadza PIN, wysyłając komunikat \textit{Insert PIN} do \textbf{ATM}.
    \item \textbf{ATM} przekazuje PIN do \textbf{Systemu Bankowego} poprzez komunikat \textit{Verify PIN}.
    \item \textbf{System Bankowy} odpowiada komunikatem \textit{PIN Verified}, potwierdzając poprawność danych uwierzytelniających.
\end{itemize}

Jeżeli uwierzytelnienie przebiegnie pomyślnie, użytkownik może kontynuować proces.

\subsection{Wybór operacji przez użytkownika}

Po poprawnej weryfikacji PIN \textbf{ATM}:
\begin{itemize}
    \item wyświetla możliwe opcje operacji \textit{Show Options},
    \item użytkownik wybiera jedną z nich \textit{Choose Option}.
\end{itemize}

Dalszy przebieg procesu zależy od wybranego wariantu, co przedstawione jest na diagramie za pomocą konstrukcji alternatywnej \texttt{alt}.

\subsection{Opcja: Sprawdzenie salda}

Jeżeli \textbf{Użytkownik} wybiera \textit{Balance Enquiry}, wykonywane są następujące kroki:
\begin{itemize}
    \item \textbf{ATM} wysyła do Systemu Bankowego żądanie \textit{Get Balance}.
    \item \textbf{System Bankowy} zwraca bieżące saldo konta (\textit{Balance}).
    \item \textbf{ATM} wyświetla saldo użytkownikowi poprzez komunikat \textit{Show Balance}.
\end{itemize}

Proces tej opcji kończy się na wyświetleniu informacji.

\subsection{Opcja: Wypłata gotówki}

Jeżeli \textbf{Użytkownik} wybiera opcję \textit{Withdraw}, proces przebiega dalej:
\begin{itemize}
    \item \textbf{ATM} prosi użytkownika o podanie kwoty do wypłaty \textit{Get Amount}.
    \item \textbf{Użytkownik} wprowadza kwotę \textit{Enter Amount}.
    \item \textbf{ATM} przesyła żądanie do Systemu Bankowego: \textit{Check Balance}.
    \item \textbf{System Bankowy} zwraca aktualne saldo (\textit{Balance}).
\end{itemize}

W tym momencie pojawia się alternatywa:
\begin{enumerate}
    \item \textbf{Saldo niewystarczające}\\
    Jeżeli saldo jest mniejsze niż żądana kwota:
    \begin{itemize}
        \item \textbf{ATM} wyświetla komunikat \textit{Insufficient Balance}.
    \end{itemize}

    \item \textbf{Saldo wystarczające}\\
    Jeżeli saldo jest wystarczające:
    \begin{itemize}
        \item \textbf{ATM} uruchamia wypłatę gotówki \textit{Dispense Cash}.
        \item \textbf{System Bankowy} aktualizuje saldo konta (\textit{Update Balance}).
        \item Po zakończeniu operacji \textbf{Użytkownik} otrzymuje komunikat: \textit{Collect Cash} i odbiera gotówkę (\textit{Collect Cash}).
    \end{itemize}
\end{enumerate}

\subsection{Opcja: Anulowanie transakcji}

Jeżeli użytkownik wybierze opcję \textit{Cancel}, ATM wyświetla komunikat \textit{Transaction Cancelled}. Proces wypłaty lub sprawdzania salda zostaje zakończony.

\subsection{Zakończenie procesu}

W końcowej części, niezależnie od wcześniejszej ścieżki:
\begin{itemize}
    \item \textbf{ATM} drukuje potwierdzenie operacji \textit{Print Receipt}.
    \item \textbf{ATM} zwraca kartę użytkownikowi \textit{Eject Card}.
    \item \textbf{Użytkownik} odbiera kartę \textit{Collect Card}.
    \item \textbf{ATM} wyświetla końcowy komunikat \textit{Thank You}, kończąc interakcję.
\end{itemize}


\section{Zadanie 2 - przykład dla czytelni.}

\begin{figure}[H]
    \centering
    \includegraphics[width=1.0\linewidth]{resources/library/sekwencja_biblioteka.png}
    \caption{Diagram sekwencji systemu czytelni przedstawiający procesy logowania, przeglądania katalogów, wypożyczania książek przez użytkownika i jak to wpływa kolejno na system, oraz proces akceptacji wypożyczenia przed administratora oraz kolejne procesy wykonywane kolejno przez system.}
    \label{fig:library_sequence_diagram}
\end{figure}

\clearpage

\subsection{Logowanie użytkownika}

Proces rozpoczyna się, gdy \textbf{Użytkownik} otwiera stronę logowania systemu bibliotecznego:
\begin{itemize}
    \item \textbf{Użytkownik} otwiera stronę logowania i wprowadza dane uwierzytelniające, wysyłając je do \textbf{Interfejsu Web}.
    \item \textbf{Interfejs Web} przekazuje dane logowania do \textbf{Systemu}.
    \item \textbf{System} przesyła dane do \textbf{Bazy danych} w celu ich weryfikacji.
    \item \textbf{Baza danych} zwraca wynik autoryzacji do \textbf{Systemu}.
\end{itemize}

Na diagramie przedstawiono alternatywę \texttt{wyniku autoryzacji}:
\begin{enumerate}
    \item \textbf{Dane poprawne} - \textbf{System} przekierowuje użytkownika do ekranu katalogu.
    \item \textbf{Dane błędne} - \textbf{Interfejs Web} wyświetla komunikat o błędzie logowania.
\end{enumerate}

\subsection{Przeglądanie katalogu}

Po poprawnym logowaniu użytkownik może rozpocząć przeglądanie dostępnych zasobów systemu czytelniczego:
\begin{itemize}
    \item \textbf{Użytkownik} wybiera katalog, wysyłając odpowiednie żądanie do \textbf{Interfejsu Web}.
    \item \textbf{Interfejs Web} przekazuje żądanie do \textbf{Systemu}.
    \item \textbf{System} pobiera listę książek z \textbf{Bazy danych}.
    \item \textbf{Baza danych} zwraca listę książek, którą \textbf{Interfejs Web} prezentuje użytkownikowi.
\end{itemize}

Następnie użytkownik może zawęzić wyniki:
\begin{itemize}
    \item \textbf{Użytkownik} ustawia filtry lub wpisuje kryteria wyszukiwania.
    \item \textbf{Interfejs Web} przekazuje filtry do \textbf{Systemu}.
    \item \textbf{System} pobiera z \textbf{Bazy danych} listę książek spełniających podane kryteria.
    \item Otrzymane dane prezentowane są użytkownikowi jako lista przefiltrowana.
\end{itemize}

\subsection{Wypożyczenie książki}

Gdy użytkownik wybierze interesującą go pozycję:
\begin{itemize}
    \item \textbf{Użytkownik} wybiera książkę i inicjuje proces wypożyczenia poprzez akcję \textit{Wypożycz}.
    \item \textbf{Interfejs Web} przekazuje żądanie wypożyczenia do \textbf{Systemu}.
    \item \textbf{System} sprawdza dostępność wybranej pozycji w \textbf{Bazie danych}.
\end{itemize}

Następnie pojawia się alternatywa \texttt{dostępność książki}:
\begin{enumerate}
    \item \textbf{Książka dostępna}:
    \begin{itemize}
        \item \textbf{System} dodaje nowe wypożyczenie ze statusem \textit{oczekujące}.
        \item \textbf{System} powiadamia \textbf{Administratora} o nowym wypożyczeniu.
        \item \textbf{Interfejs Web} wyświetla komunikat: \textit{Wypożyczenie oczekuje na akceptację}.
    \end{itemize}

    \item \textbf{Książka niedostępna}:
    \begin{itemize}
        \item \textbf{Interfejs Web} prezentuje użytkownikowi komunikat: \textit{Książka niedostępna}.
    \end{itemize}
\end{enumerate}

\subsection{Akceptacja wypożyczenia}

Proces kończy się po stronie administracyjnej:
\begin{itemize}
    \item \textbf{Administrator} akceptuje wypożyczenie w systemie.
    \item \textbf{System} aktualizuje status wypożyczenia na \textit{Wypożyczona} w \textbf{Bazie danych}.
    \item \textbf{System} wysyła powiadomienie do \textbf{Użytkownika} o zaakceptowaniu wypożyczenia.
    \newline\newline
\end{itemize}


\section{Zadanie 3 - przykład dla projektu zespołowego.}

\newpage
\begin{figure}[H]
    \centering
    \includegraphics[width=1.0\linewidth]{resources/kairo/top_half.png}
    \label{fig:kairo_sequence_diagram_top}
\end{figure}
\begin{figure}[H]
    \centering
    \includegraphics[width=1.0\linewidth]{resources/kairo/bottom_half.png}
    \label{fig:kairo_sequence_diagram_bottom}
\end{figure}

\subsection{Rejestracja użytkownika}

Proces rozpoczyna się, gdy \textbf{Użytkownik} otwiera ekran rejestracji w aplikacji mobilnej:
\begin{itemize}
    \item \textbf{Użytkownik} otwiera formularz rejestracyjny i wypełnia wymagane pola.
    \item Aplikacja mobilna (\textbf{Frontend}) wysyła dane rejestracyjne do \textbf{Backendu} poprzez żądanie \texttt{POST /register}.
    \item \textbf{Backend} sprawdza w \textbf{Bazie danych}, czy podany adres e-mail już istnieje.
    \item \textbf{Baza danych} zwraca wynik weryfikacji.
\end{itemize}

Występuje tu konstrukcja alternatywna \texttt{wyniku sprawdzania}:
\begin{enumerate}
    \item \textbf{Email wolny}:  
    \begin{itemize}
        \item \textbf{Backend} tworzy nowy rekord użytkownika w tabeli \texttt{users}.
        \item \textbf{Baza danych} potwierdza poprawne dodanie użytkownika.
        \item \textbf{Frontend} otrzymuje odpowiedź \texttt{201 Created}, informując o utworzonym koncie użytkownika.
    \end{itemize}

    \item \textbf{Email zajęty}:
    \begin{itemize}
        \item \textbf{Backend} zwraca komunikat błędu \texttt{409 Conflict}.
        \item \textbf{Frontend} informuje użytkownika o duplikacie adresu e-mail.
    \end{itemize}
\end{enumerate}

\subsection{Logowanie użytkownika}

Aby uzyskać dostęp do aplikacji, użytkownik wykonuje:
\begin{itemize}
    \item Otwarcie ekranu logowania i wprowadzenie danych.
    \item \textbf{Frontend} wysyła żądanie \texttt{POST /login}.
    \item \textbf{Backend} pobiera dane użytkownika z \textbf{Bazy danych}.
\end{itemize}

Następnie występuje alternatywa:
\begin{enumerate}
    \item \textbf{Hasło poprawne}:
    \begin{itemize}
        \item \textbf{Backend} zwraca kod \texttt{200 OK} wraz z tokenem sesyjnym.
        \item Aplikacja informuje o poprawnym logowaniu.
    \end{itemize}

    \item \textbf{Hasło błędne}:
    \begin{itemize}
        \item \textbf{Backend} zwraca kod \texttt{401 Unauthorized}.
        \item \textbf{Frontend} wyświetla komunikat o błędzie.
    \end{itemize}
\end{enumerate}

\subsection{Dodawanie nowego nawyku z przypomnieniem}

Po zalogowaniu użytkownik może stworzyć nowy nawyk:
\begin{itemize}
    \item \textbf{Użytkownik} uzupełnia formularz nowego nawyku i opcjonalne przypomnienie.
    \item \textbf{Frontend} wysyła żądanie \texttt{POST /habits}.
    \item \textbf{Backend} dodaje rekordy do tabel \texttt{habits} oraz \texttt{reminders} w \textbf{Bazie danych}.
    \item \textbf{System powiadomień} rejestruje nowe przypomnienie.
    \item Aplikacja otrzymuje odpowiedź \texttt{201 Created}.
\end{itemize}

\subsection{Przeglądanie i edycja nawyków}

Użytkownik może przeglądać lub modyfikować swoje nawyki:
\begin{itemize}
    \item Żądanie \texttt{GET /habits} trafia do \textbf{Backendu}.
    \item \textbf{Backend} pobiera listę nawyków z \textbf{Bazy danych}.
    \item \textbf{Frontend} wyświetla pobraną listę.
\end{itemize}

W przypadku edycji lub usunięcia:
\begin{itemize}
    \item \textbf{Frontend} wysyła żądanie \texttt{PUT/DELETE /habits/\{id\}}.
    \item \textbf{Backend} dokonuje odpowiedniej operacji w \textbf{Bazie danych}.
    \item Po zakończeniu przetwarzania zwracany jest kod \texttt{200 OK}.
\end{itemize}

\subsection{Przeglądanie postępów}

Użytkownik może przeanalizować swoje statystyki:
\begin{itemize}
    \item \textbf{Frontend} wykonuje żądanie \texttt{GET /progress}.
    \item \textbf{Backend} pobiera dane statystyczne z \textbf{Bazy danych}.
    \item Aplikacja prezentuje użytkownikowi wykresy, streaki i osiągnięcia.
\end{itemize}

\subsection{Dodawanie znajomego}

W celu zwiększenia motywacji użytkownik może dodać znajomego:
\begin{itemize}
    \item \textbf{Frontend} wysyła żądanie \texttt{POST /friends/invite}.
    \item \textbf{Backend} sprawdza istnienie użytkownika i dodaje relację w tabeli \texttt{friendships}.
    \item \textbf{Frontend} otrzymuje odpowiedź \texttt{201 Created}.
\end{itemize}

\subsection{Pingowanie znajomego}

Użytkownik może wysłać szybkie przypomnienie do znajomego:
\begin{itemize}
    \item Wysyłane jest żądanie \texttt{POST /friends/ping}.
    \item \textbf{Backend} inicjuje powiadomienie w \textbf{Systemie powiadomień}.
    \item \textbf{Frontend} otrzymuje odpowiedź \texttt{200 OK}.
\end{itemize}

\subsection{Zarządzanie powiadomieniami}

Użytkownik może dostosować parametry przypomnień:
\begin{itemize}
    \item \textbf{Frontend} wysyła żądanie \texttt{PUT /settings/notifications}.
    \item \textbf{Backend} aktualizuje ustawienia w \textbf{Bazie danych}.
    \item Zwracana jest odpowiedź \texttt{200 OK}.
\end{itemize}

\subsection{Personalizacja motywu aplikacji}

Użytkownik może zmieniać motyw wizualny:
\begin{itemize}
    \item Żądanie \texttt{PUT /settings/theme} trafia do \textbf{Backendu}.
    \item \textbf{Baza danych} aktualizuje preferencje.
    \item Aplikacja otrzymuje odpowiedź \texttt{200 OK}.
\end{itemize}

\subsection{Przypomnienia (automatyczne)}

System obsługuje wysyłanie przypomnień nawet bez akcji użytkownika:
\begin{itemize}
    \item \textbf{System powiadomień} cyklicznie sprawdza ustawienia i czas przypomnień.
    \item W przypadku spełnienia warunków wysyłane jest powiadomienie push.
    \item \textbf{Backend} pobiera dane użytkownika i nawyku celem generacji treści powiadomienia.
    \item \textbf{Frontend} prezentuje powiadomienie użytkownikowi.
\end{itemize}

\section{Diagram interakcji sekwencji i współpracy - omówienie różnic}

Diagramy interakcji w notacji UML można podzielić na dwa główne typy: diagram sekwencji oraz diagram współpracy (zwany również diagramem komunikacji). Oba typy modelują sposób, w jaki obiekty wzajemnie oddziałują, jednak każdy z nich eksponuje inne aspekty analizowanego procesu. W niniejszej sekcji omówiono kluczowe cechy obu diagramów oraz wskazano fundamentalne różnice między nimi.

\subsection{Charakterystyka diagramu sekwencji}
\begin{itemize}
    \item Koncentruje się na przebiegu zdarzeń w czasie, ukazując chronologiczną kolejność komunikatów.
    \item Wykorzystuje pionowe linie życia, które wizualizują czas istnienia obiektów w trakcie interakcji.
    \item Komunikaty ułożone są liniowo od góry do dołu, co pozwala prześledzić pełen przebieg procesu.
    \item Podkreśla rytm, czas trwania i zależności czasowe między operacjami.
    \item Jest szczególnie przydatny w analizie procesów logicznych i dynamicznych przepływów danych.
\end{itemize}

\subsection{Charakterystyka diagramu współpracy}
\begin{itemize}
    \item Skupia się na strukturze komunikacji między obiektami, a nie na czasie wykonania.
    \item Komunikaty opisane są numeracją odzwierciedlającą kolejność operacji, ale nie ich czasową liniowość.
    \item Umożliwia analizę organizacji i sieci powiązań między obiektami.
    \item Przedstawia obiekty w układzie przestrzennym, co pozwala ocenić ich relacje oraz stopień zależności.
    \item Często stosowany podczas projektowania architektury systemu lub sieci współpracujących agentów.
\end{itemize}

\subsection{Porównanie obu typów diagramów}
\begin{itemize}
    \item \textbf{Perspektywa czasu:}  
    Diagram sekwencji akcentuje oś czasu, natomiast diagram współpracy przedstawia relacje strukturalne, bez dokładnego odwzorowania przebiegu czasowego.
    \item \textbf{Priorytet wizualizacji:}  
    Diagram sekwencji eksponuje kolejność wywołań, a diagram współpracy - zależności między obiektami.
    \item \textbf{Złożoność interpretacji:}  
    Diagram sekwencji jest łatwiejszy do odczytania w przypadku długich procesów. Diagram współpracy lepiej prezentuje rozbudowane relacje między obiektami.
    \item \textbf{Różne zastosowania projektowe:}  
    Diagram sekwencji zalecany jest dla analizy procesów użytkownika (use-case), a diagram współpracy dla projektowania architektury i struktury systemu.
\end{itemize}

\subsection{Kiedy stosować każdy z diagramów}
\begin{itemize}
    \item Diagram sekwencji:  
    \begin{itemize}
        \item analizowanie przepływów czasowych i zależności operacji,
        \item wykrywanie błędów w logicznym następstwie czynności,
        \item modelowanie procesów użytkownika w systemach interaktywnych,
        \item analiza zachowania systemów czasu rzeczywistego.
    \end{itemize}
    \item Diagram współpracy:
    \begin{itemize}
        \item analizowanie zależności obiektów w architekturze systemu,
        \item projektowanie komunikujących się modułów,
        \item optymalizacja struktury interakcji,
        \item modelowanie systemów wieloagentowych.
    \end{itemize}
\end{itemize}

\section{Wnioski}

Diagramy sekwencji i współpracy stanowią kluczowe narzędzia w inżynierii oprogramowania, umożliwiając zrozumienie dynamiki działania systemu oraz relacji między jego elementami. Ich stosowanie przynosi liczne korzyści na etapie projektowania, dokumentowania i analizy procesów informatycznych.

\subsection{Zastosowania diagramów interakcji}
\begin{itemize}
    \item Dokumentacja zachowania systemów w czasie rzeczywistym.
    \item Analiza funkcjonalności z poziomu użytkownika (modelowanie przypadków użycia).
    \item Projektowanie interakcji między modułami i podsystemami.
    \item Wspieranie analizy wymagań oraz walidacja kompletności specyfikacji.
\end{itemize}

\subsection{Celowość ich wykorzystania w procesie informatycznym}
\begin{itemize}
    \item Pozwalają wcześnie wykryć nieścisłości w projektowaniu logiki systemu.
    \item Stanowią podstawę do tworzenia kodu źródłowego i architektury systemowej.
    \item Zapewniają formalny sposób komunikacji pomiędzy członkami zespołu projektowego.
    \item Ułatwiają analizę i optymalizację przepływu danych i komunikatów.
\end{itemize}

\subsection{Znaczenie dla bezpieczeństwa i niezawodności systemu}
\begin{itemize}
    \item Umożliwiają identyfikację potencjalnych punktów awarii lub przeciążenia.
    \item Wspierają analizę ryzyka poprzez modelowanie scenariuszy wyjątkowych.
    \item Pozwalają przewidzieć konsekwencje błędnych interakcji między komponentami.
\end{itemize}

\subsection{Rola w procesie implementacji i testowania}
\begin{itemize}
    \item Ułatwiają projektowanie struktur klas i interfejsów API.
    \item Tworzą podstawę dla testów jednostkowych i integracyjnych.
    \item Pozwalają zasymulować różne ścieżki wykonywania programu.
    \item Wspomagają automatyzację testów poprzez mapowanie scenariuszy.
\end{itemize}

\subsection{Podsumowanie wniosków}
\begin{itemize}
    \item Diagramy interakcji są niezbędne w procesie projektowania dynamicznego zachowania systemu.
    \item Zapewniają pełną przejrzystość procesów oraz hierarchii komunikatów.
    \item Poprawiają efektywność projektowania, implementacji i weryfikacji oprogramowania.
    \item Ich stosowanie przekłada się na wyższą jakość, stabilność i zrozumiałość systemu informatycznego.
\end{itemize}

\end{document}

