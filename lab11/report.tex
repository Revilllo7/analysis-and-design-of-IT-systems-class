\documentclass[12pt,a4paper]{article}

% ------------------ Pakiety ------------------
\usepackage[utf8]{inputenc}
\usepackage[T1]{fontenc}
\usepackage[polish]{babel}
\usepackage{geometry}
\usepackage{amsmath}
\usepackage{graphicx}
\usepackage{hyperref}
\usepackage{lmodern}
\usepackage{setspace}
\usepackage{titlesec}
\usepackage{array}
\usepackage{float}
\usepackage{listings}
\usepackage{color}
\usepackage{verbatim}

% ------------------ Ustawienia ------------------
\geometry{margin=2.5cm}
\setstretch{1.3}
\titleformat{\section}{\large\bfseries}{\thesection.}{1em}{}
\titleformat{\subsection}{\normalsize\bfseries}{\thesubsection.}{1em}{}
\definecolor{lightgray}{rgb}{.9,.9,.9}
\definecolor{darkgray}{rgb}{.4,.4,.4}
\definecolor{purple}{rgb}{0.65, 0.12, 0.82}

\lstdefinelanguage{JavaScript}{
  keywords={typeof, new, true, false, catch, function, return, null, catch, switch, var, if, in, while, do, else, case, break},
  keywordstyle=\color{blue}\bfseries,
  ndkeywords={class, export, boolean, throw, implements, import, this},
  ndkeywordstyle=\color{darkgray}\bfseries,
  identifierstyle=\color{black},
  sensitive=false,
  comment=[l]{//},
  morecomment=[s]{/*}{*/},
  commentstyle=\color{purple}\ttfamily,
  stringstyle=\color{red}\ttfamily,
  morestring=[b]',
  morestring=[b]''
}

\lstset{
   language=JavaScript,
   backgroundcolor=\color{lightgray},
   extendedchars=true,
   basicstyle=\footnotesize\ttfamily,
   showstringspaces=false,
   showspaces=false,
   numbers=left,
   numberstyle=\footnotesize,
   numbersep=9pt,
   tabsize=2,
   breaklines=true,
   showtabs=false,
   captionpos=b
}

\DeclareUnicodeCharacter{25CF}{$\bullet$}
\DeclareUnicodeCharacter{251C}{\mbox{\kern.23em
  \vrule height2.2exdepth1exwidth.4pt\vrule height2.2ptdepth-1.8ptwidth.23em}}
\DeclareUnicodeCharacter{2500}{\mbox{\vrule height2.2ptdepth-1.8ptwidth.5em}}
\DeclareUnicodeCharacter{2514}{\mbox{\kern.23em
  \vrule height2.2exdepth-1.8ptwidth.4pt\vrule height2.2ptdepth-1.8ptwidth.23em}}

\hypersetup{
    colorlinks=true,
    linkcolor=black,
    urlcolor=blue,
    pdftitle={Diagramy komponentów i wdrożenia w procesie projektowym}
}

% ------------------ Dane ------------------

\title{Uniwersytet Gdański
Wydział Matematyki, Fizyki i Informatyki
Instytut Informatyki}
\author{Oliver Gruba, Maciej Nasiadka}

\begin{document}
\maketitle
\begin{table}
    \centering
    \begin{tabular}{|>{\raggedright\arraybackslash}p{0.5\linewidth}|>{\raggedright\arraybackslash}p{0.4\linewidth}|}\hline
         Imię i Nazwisko (nr indeksu)& Oliver Gruba (292583) \\
         & Maciej Nasiadka (292574)\\\hline
         Nazwa uczelni& Uniwersytet Gdański\\\hline
         Kierunek& Informatyka (profil praktyczny)\\\hline
         Prowadzący& dr inż. Stanisław Witkowski\\\hline
         Nazwa ćwiczenia& Diagramy komponentów i wdrożenia w procesie projektowym\\\hline
         Numer sprawozdania& 8\\\hline
         Data zajęć& 11.12.2025\\\hline
         Data oddania& 17.12.2025\\\hline
         Miejsce na ocenę& \\ \hline
    \end{tabular}
\end{table}
\clearpage

% ------------------ Dokument ------------------

\tableofcontents

\clearpage

% -------------------------------------------------------------

\section{Wprowadzenie do modelowania architektury systemów informatycznych}

Modelowanie architektury systemu odgrywa kluczową rolę w procesie projektowym, ponieważ umożliwia przejrzyste przedstawienie struktury aplikacji, zależności między jej elementami oraz sposobu, w jaki system zostanie umieszczony w środowisku sprzętowo–sieciowym. W tym celu wykorzystuje się dwa komplementarne diagramy: \textbf{diagram komponentów} oraz \textbf{diagram wdrożenia}.

Diagramy te stanowią istotny element dokumentacji technicznej, wspierają komunikację pomiędzy zespołami projektowymi oraz ułatwiają analizę możliwości rozwoju i skalowalności systemu.

% -------------------------------------------------------------

\section{Diagram komponentów}

\subsection{Charakterystyka i przeznaczenie}

Diagram komponentów w notacji UML przedstawia logiczną strukturę systemu poprzez identyfikację jego głównych modułów oraz relacji między nimi. Komponent definiowany jest jako niezależna jednostka o określonym interfejsie, która może być rozwijana, testowana i wdrażana autonomicznie.

Celem diagramu komponentów jest:
\begin{itemize}
    \item wyodrębnienie części systemu odpowiedzialnych za konkretne funkcjonalności,
    \item zdefiniowanie interfejsów udostępnianych na zewnątrz,
    \item ukazanie zależności logicznych między modułami,
    \item wsparcie projektowania architektury warstwowej.
\end{itemize}

\subsection{Poziomy szczegółowości diagramu komponentów}

Diagram komponentów może być tworzony na różnych poziomach abstrakcji, w zależności od etapu projektu oraz potrzeb odbiorców dokumentacji.

\begin{itemize}
    \item \textbf{Poziom koncepcyjny} – przedstawia główne moduły systemu bez wchodzenia w szczegóły implementacyjne; stosowany na wczesnym etapie analizy.
    \item \textbf{Poziom logiczny} – uwzględnia interfejsy, zależności oraz podział na warstwy architektury.
    \item \textbf{Poziom techniczny} – zawiera konkretne technologie, biblioteki oraz artefakty programistyczne.
\end{itemize}

Dobór odpowiedniego poziomu szczegółowości wpływa na czytelność diagramu oraz jego użyteczność w procesie decyzyjnym.


\subsection{Podstawowe elementy diagramu komponentów}

\subsubsection{Komponenty}

\begin{figure}[H]
    \centering
    \includegraphics[width=0.65\linewidth]{resources/component_diagram/komponenty podstawy.png}
    \caption{Graficzne przedstawienie podstawowych elementów diagramu komponentów}
    \label{fig:podstawy}
\end{figure}

Są to fizyczne lub logiczne moduły systemu, oznaczane prostokątem z symbolem `<<component>>`. Mogą reprezentować:
\begin{itemize}
    \item moduły aplikacji (np. moduł raportowania),
    \item biblioteki,
    \item usługi (API),
    \item warstwy logiki biznesowej.
\end{itemize}

\subsubsection{Interfejsy}
Interfejsy definiują punkty komunikacji między komponentami. Dzielą się na:
\begin{itemize}
    \item \textbf{interfejsy dostarczane (provided)} – funkcje, które komponent udostępnia,
    \item \textbf{interfejsy wymagane (required)} – funkcje niezbędne komponentowi do działania.
\end{itemize}

\subsubsection{Zależności}
Zależności przedstawiają kierunek, w którym komponent korzysta z interfejsu innego komponentu.

\begin{figure}[H]
    \centering
    \includegraphics[width=0.65\linewidth]{resources/component_diagram/komponenty powiązania i relacje.png}
    \caption{Graficzne przedstawienie powiązania i relacji diagramu komponentów}
    \label{fig:powiązania}
\end{figure}

\subsubsection{Pakiety i grupowanie komponentów}

W celu zwiększenia czytelności diagramu komponentów stosuje się pakiety (packages), które pozwalają logicznie grupować powiązane elementy systemu.

Pakiety umożliwiają:
\begin{itemize}
    \item odzwierciedlenie architektury warstwowej (np. prezentacja, logika biznesowa, dane),
    \item ograniczenie liczby widocznych zależności,
    \item lepszą organizację dużych i złożonych systemów.
\end{itemize}

Grupowanie komponentów sprzyja również podziałowi pracy pomiędzy zespoły projektowe.


\subsubsection{Przepływ danych}

\begin{figure}[H]
    \centering
    \includegraphics[width=0.65\linewidth]{resources/component_diagram/komponenty przepływ danych.png}
    \caption{Graficzne przedstawienie przepływu danych w  diagramie komponentów}
    \label{fig:przepływ}
\end{figure}

\subsection{Diagram komponentów a architektura systemu}

Diagram komponentów jest bezpośrednio powiązany z wybraną architekturą systemu, taką jak:
\begin{itemize}
    \item architektura warstwowa,
    \item architektura klient–serwer,
    \item architektura mikroserwisowa,
    \item architektura oparta na usługach (SOA).
\end{itemize}

Poprawnie zaprojektowany diagram komponentów pozwala ocenić stopień luźnego powiązania (low coupling) oraz spójności modułów (high cohesion), co ma kluczowe znaczenie dla utrzymania i rozwoju systemu.


\subsection{Znaczenie diagramu komponentów w projektowaniu}

Diagram komponentów wspiera projektowanie architektury modularnej oraz pozwala na:
\begin{itemize}
    \item planowanie zespołowej pracy nad modułami,
    \item ograniczanie ryzyka integracji,
    \item przygotowanie systemu pod mikroserwisy lub architekturę rozproszoną,
    \item oszacowanie punktów krytycznych systemu.
\end{itemize}

% -------------------------------------------------------------

\section{Diagram wdrożenia}

\subsection{Rola diagramu wdrożenia}

Diagram wdrożenia (Deployment Diagram) ukazuje fizyczne rozmieszczenie elementów systemu w środowisku sprzętowym. Przedstawia sposób uruchamiania komponentów na serwerach, urządzeniach sieciowych lub maszynach wirtualnych. Wskazuje również połączenia komunikacyjne oraz zależności infrastrukturalne.

Diagram ten jest kluczowy w projektach wymagających:
\begin{itemize}
    \item skalowalności,
    \item rozproszenia usług,
    \item integracji systemów,
    \item specyficznych wymogów wydajnościowych.
\end{itemize}

\subsection{Diagram wdrożenia w cyklu życia systemu}

Diagram wdrożenia znajduje zastosowanie na wielu etapach cyklu życia systemu informatycznego:
\begin{itemize}
    \item podczas projektowania infrastruktury,
    \item w fazie wdrożeniowej i testowej,
    \item przy planowaniu skalowania systemu,
    \item w procesie utrzymania i monitorowania.
\end{itemize}

Stanowi on istotne źródło informacji dla administratorów systemów oraz zespołów odpowiedzialnych za utrzymanie środowisk produkcyjnych.


\subsection{Elementy diagramu wdrożenia}

\begin{figure}[H]
    \centering
    \includegraphics[width=0.65\linewidth]{resources/deployment_diagram/wdrożenia podstawy.png}
    \caption{Graficzne przedstawienie podstawowych elementów diagramu wdrożenia}
    \label{fig:wdrożenia}
\end{figure}

\subsubsection{Węzły (Nodes)}
Węzły reprezentują fizyczne lub wirtualne zasoby obliczeniowe, np.:
\begin{itemize}
    \item serwer aplikacyjny,
    \item urządzenie mobilne,
    \item baza danych,
    \item chmura obliczeniowa.
\end{itemize}

\subsubsection{Środowiska wdrożeniowe}

Diagram wdrożenia może uwzględniać różne środowiska, w których funkcjonuje system:
\begin{itemize}
    \item środowisko deweloperskie (DEV),
    \item środowisko testowe (TEST),
    \item środowisko produkcyjne (PROD).
\end{itemize}

Rozróżnienie środowisk pozwala na analizę różnic infrastrukturalnych oraz minimalizację ryzyka błędów podczas wdrażania nowych wersji systemu.


\subsubsection{Artefakty}
Artefakty oznaczają rzeczywiste pliki wdrożeniowe lub procesy uruchomieniowe, takie jak:
\begin{itemize}
    \item pliki JAR, WAR,
    \item kontenery Docker,
    \item pliki konfiguracyjne.
\end{itemize}

\subsubsection{Połączenia komunikacyjne}
Określają sposób komunikacji między węzłami – mogą reprezentować protokoły, porty, szyfrowanie.

\subsection{Znaczenie diagramu wdrożenia w analizie architektury}

Diagram wdrożenia umożliwia:
\begin{itemize}
    \item identyfikację potencjalnych wąskich gardeł,
    \item planowanie pojemności infrastruktury,
    \item przygotowanie systemu pod dostępność 24/7,
    \item analizę bezpieczeństwa i redundancji.
\end{itemize}

% -------------------------------------------------------------

\section{Powiązanie diagramów komponentów i diagramów wdrożenia}

Diagramy te są ze sobą ściśle powiązane:
\begin{itemize}
    \item diagram komponentów określa \textbf{co} tworzy system,
    \item diagram wdrożenia określa \textbf{gdzie} te elementy są uruchamiane.
\end{itemize}

W procesie projektowym tworzy się najpierw strukturę logiczną, a następnie odwzorowuje ją w środowisku produkcyjnym.

Zależności te wspierają:
\begin{itemize}
    \item planowanie migracji do chmury,
    \item analizę wydajności,
    \item poprawne rozmieszczenie zależności sieciowych,
    \item rozdzielenie odpowiedzialności pomiędzy zespoły DevOps i programistyczne.
\end{itemize}

\subsection{Spójność logiczna i fizyczna architektury}

Zachowanie spójności pomiędzy diagramem komponentów a diagramem wdrożenia jest kluczowe dla poprawnej realizacji projektu.

Każdy komponent logiczny powinien mieć swoje jednoznaczne odwzorowanie w diagramie wdrożenia, co umożliwia:
\begin{itemize}
    \item kontrolę kompletności architektury,
    \item identyfikację brakujących elementów infrastruktury,
    \item analizę wpływu zmian logicznych na środowisko fizyczne.
\end{itemize}


% -------------------------------------------------------------

\newpage
\section{Przykładowe diagramy}

\subsection{Przykładowy diagram komponentów}

\begin{figure}[H]
    \centering
    \includegraphics[width=0.9\linewidth]{resources/diagram_komponentow.png}
\end{figure}

Diagram komponentów przedstawia logiczną strukturę systemu, pokazując główne moduły (komponenty), ich interfejsy oraz zależności pomiędzy nimi. Ułatwia zrozumienie podziału odpowiedzialności oraz komunikacji między częściami systemu.

\begin{itemize}
    \item \textbf{MobileApp} i \textbf{WebApp} -- dwa różne frontendy korzystające z tych samych interfejsów API.
    \item \textbf{IUserAPI}, \textbf{IAuthService}, \textbf{IDatabase} -- interfejsy definiujące punkty komunikacji pomiędzy komponentami.
    \item \textbf{Auth Service} -- odpowiada za uwierzytelnianie, komunikuje się z API server przez interfejs IAuthService.
    \item \textbf{API server} -- centralny komponent obsługujący logikę biznesową, komunikuje się z bazą danych (IDatabase) oraz obsługuje żądania REST od aplikacji frontendowych.
    \item \textbf{Database} -- przechowuje dane, dostępna przez interfejs IDatabase.
    \item Połączenia REST i Auth są wyraźnie oznaczone, co ułatwia analizę przepływu danych i bezpieczeństwa.
\end{itemize}

\newpage

\subsection{Przykładowy diagram wdrożenia}

\begin{figure}[H]
    \centering
    \includegraphics[width=1.0\linewidth]{resources/diagram_wdrozenia.png}
\end{figure}

Diagram wdrożenia przedstawia fizyczną architekturę systemu informatycznego, pokazując rozmieszczenie komponentów na serwerach oraz sposób komunikacji między nimi. Umożliwia zrozumienie, jak poszczególne elementy systemu są rozmieszczone w infrastrukturze oraz jakie protokoły i połączenia są wykorzystywane.

\begin{itemize}
    \item \textbf{Load Balancer} -- rozdziela ruch HTTP/HTTPS do dwóch serwerów aplikacji, zapewniając równoważenie obciążenia i wysoką dostępność.
    \item \textbf{Serwer Aplikacji 1 i 2} -- obsługują aplikację webową, komunikują się z serwerem plików (protokół SMB/NFS) oraz z serwerem bazy danych (JDBC/SQL).
    \item \textbf{Serwer Plików} -- przechowuje zasoby statyczne, dostępny dla obu serwerów aplikacji.
    \item \textbf{Serwer Bazy Danych} -- przechowuje dane aplikacji, dostępny dla obu serwerów aplikacji.
    \item Komunikacja między serwerami jest jasno oznaczona (HTTP/HTTPS, SMB/NFS, JDBC/SQL), co ułatwia analizę bezpieczeństwa i wydajności.
\end{itemize}


% -------------------------------------------------------------

\section{Znaczenie diagramów w procesie projektowym}

Diagramy komponentów i wdrożenia dostarczają architektom, projektantom i zespołom deweloperskim kluczowych informacji o budowie systemu. Ich wykorzystanie prowadzi do:

\subsection{Poprawy jakości projektu}

\begin{itemize}
    \item umożliwiają wykrycie błędów strukturalnych na wczesnym etapie,
    \item pozwalają na spójną organizację warstw systemu,
    \item sprzyjają modularności i testowalności kodu.
\end{itemize}

\subsection{Wsparcia integracji i wdrożeń}

\begin{itemize}
    \item ułatwiają komunikację między zespołami,
    \item dostarczają pełnego obrazu zależności logistycznych i sprzętowych,
    \item wspierają proces przygotowania środowisk (DEV, TEST, PROD).
\end{itemize}

\subsection{Zwiększenia przejrzystości}

\begin{itemize}
    \item umożliwiają analizę skalowalności,
    \item przedstawiają obszary podatne na awarie,
    \item pozwalają na planowanie zasobów infrastrukturalnych.
\end{itemize}

\section{Najczęstsze błędy w projektowaniu diagramów}

Podczas tworzenia diagramów komponentów i wdrożenia często pojawiają się błędy, takie jak:
\begin{itemize}
    \item zbyt niski lub zbyt wysoki poziom szczegółowości,
    \item mieszanie aspektów logicznych i fizycznych w jednym diagramie,
    \item brak jednoznacznych interfejsów,
    \item pomijanie połączeń komunikacyjnych i zależności infrastrukturalnych.
\end{itemize}

Świadomość tych problemów pozwala tworzyć bardziej czytelną i użyteczną dokumentację architektoniczną.


% -------------------------------------------------------------

\section{Wnioski}

Analiza diagramów komponentów oraz diagramów wdrożenia pokazuje, że oba te narzędzia pełnią komplementarne role w procesie projektowania architektury systemów informatycznych. Ich wspólne wykorzystanie umożliwia uzyskanie spójnego i pełnego obrazu struktury systemu – zarówno na poziomie logiki, jak i fizycznego rozmieszczenia elementów. Poniższe subsekcje przedstawiają najważniejsze konkluzje wynikające z przeprowadzonego opracowania.

\subsection{Wnioski dotyczące diagramu komponentów}

Diagram komponentów stanowi fundament projektowania architektury logicznej systemu. Na podstawie analizy można wskazać następujące kluczowe wnioski:

\begin{itemize}
\item Diagram ten umożliwia jasne wyodrębnienie modułów systemu oraz ich odpowiedzialności, co zwiększa spójność projektową.
\item Interfejsy (provided/required) stanowią przejrzysty sposób definiowania komunikacji pomiędzy modułami, ograniczając ryzyko niejednoznaczności w implementacji.
\item Reprezentacja zależności pozwala na wczesne wykrycie potencjalnych problemów integracyjnych oraz miejsc, które mogą stać się wąskimi gardłami.
\item Diagram komponentów wspiera podejście warstwowe, ułatwiając rozdzielanie odpowiedzialności i zwiększając modularność systemu.
\item Przedstawiona struktura logiczna jest kluczowa przy planowaniu przyszłego rozwoju systemu oraz przy przechodzeniu w kierunku architektury rozproszonej (w tym mikroserwisów).
\end{itemize}

\subsection{Wnioski dotyczące diagramu wdrożenia}

Diagram wdrożenia dostarcza wiedzy o faktycznym rozmieszczeniu systemu w środowisku sprzętowym i sieciowym. Najważniejsze obserwacje to:

\begin{itemize}
\item Diagram wdrożenia ułatwia ocenę zapotrzebowania na zasoby infrastrukturalne, takie jak moc obliczeniowa, pamięć czy przepustowość łączy.
\item Graficzna reprezentacja węzłów i artefaktów pozwala precyzyjnie określić, gdzie działają poszczególne komponenty systemu.
\item Diagram ten jest niezbędny przy analizie skalowalności, bezpieczeństwa, redundancji oraz wysokiej dostępności systemu.
\item Jasne przedstawienie połączeń komunikacyjnych umożliwia identyfikację potencjalnych miejsc awarii oraz planowanie mechanizmów ochronnych.
\item Dokumentacja wdrożeniowa wspiera zespoły DevOps w konfiguracji środowisk oraz automatyzacji procesów CI/CD.
\end{itemize}

\subsection{Wnioski dotyczące wzajemnych powiązań diagramów}

Połączenie diagramu komponentów i diagramu wdrożenia pozwala na stworzenie pełnej, wielopoziomowej dokumentacji architektury systemu. Z tej relacji wynikają następujące wnioski:

\begin{itemize}
\item Diagram komponentów określa funkcjonalne elementy systemu, natomiast diagram wdrożenia pokazuje ich fizyczne umiejscowienie – oba są niezbędne do zrozumienia całości architektury.
\item Powiązanie tych diagramów ułatwia planowanie migracji systemu do chmury lub do środowiska rozproszonego.
\item Wspólna analiza komponentów i wdrożenia pozwala zidentyfikować zależności sieciowe, opóźnienia komunikacyjne oraz wymagania dotyczące integracji.
\item Dzięki połączeniu obu modeli możliwe jest lepsze rozdzielenie obowiązków pomiędzy zespoły projektowe i operacyjne.
\item Spójna dokumentacja architektury przyczynia się do poprawy jakości całego projektu oraz ogranicza ryzyko błędów na etapie implementacji i wdrożeń.
\end{itemize}

\subsection{Podsumowanie ogólne}

Zastosowanie diagramów komponentów i wdrożenia stanowi ważny element profesjonalnego projektowania systemów informatycznych. Pozwala na:

\begin{itemize}
\item zrozumienie konstrukcji systemu na poziomie logicznym i fizycznym,
\item usprawnienie komunikacji między zespołami technicznymi,
\item wczesne wykrycie problemów oraz racjonalne planowanie rozwoju,
\item przygotowanie architektury na przyszłą skalowalność i automatyzację wdrożeń.
\end{itemize}

Wspólne wykorzystanie obu diagramów tworzy solidną podstawę dla dalszych etapów projektowania, implementacji i utrzymania systemu.

\end{document}
